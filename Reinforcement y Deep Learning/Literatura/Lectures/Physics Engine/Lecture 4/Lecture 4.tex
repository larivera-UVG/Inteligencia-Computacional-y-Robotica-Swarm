\documentclass{beamer}
\usetheme{Warsaw}
\usepackage{nhtvslides}
\usepackage{graphicx}
\usepackage{listings}
\lstset{language=CAML,
basicstyle=\ttfamily\footnotesize,
frame=shadowbox,
breaklines=true}
\usepackage[utf8]{inputenc}

\title{Building a physics engine - part 1: architecture}

\author{Dr. Giuseppe Maggiore}

\institute{NHTV University of Applied Sciences \\ 
Breda, Netherlands}

\date{}

\begin{document}
\maketitle

\section{Physics engine basics}
\begin{slide}{Physics engine basics}{Main goals}{
\item Apply Newton's law of motion $F = ma$ to linear and rotational movement (easy)
\item Ensure no interpenetration of moving objects (absolutely \textbf{not easy})
}\end{slide}

\begin{slide}{Physics engine basics}{Interpenetration avoidance}{
\item Find pairs of touching objects
\item Find time of contact
\item Find points of contact
\item Apply forces to ensure no interpenetration/stacking/etc.
}\end{slide}

\begin{slide}{Physics engine basics}{Finding colliding pairs}{
\item Broad-phase collision detection with AABBs
\item Medium-phase collision detection with bounding volumes (BS, OBB)
\item Narrow-phase collision detection to find exact time and points of contact
}\end{slide}

\begin{slide}{Physics engine basics}{Collision response}{
\item Between pairs of objects: easy, just apply the laws of conservation of momentum
\item How about between multiple objects, maybe even stacked?
\item Solving in pairs does not work!
\item We need to solve all of these constraints at the same time
}\end{slide}

\begin{slide}{Physics engine basics}{Collision response and constraints}{
\item Collision response balances external forces and velocities; it applies to multiple kinds of constraints:
\begin{itemize}
\item Contact constraints (the obvious ones)
\item Distance constraints
\item Friction constraints
\item ...
\end{itemize}
}\end{slide}

\begin{frame}[fragile]{General layout of a physics engine}
\begin{lstlisting}
physics.Initialize:
  setup AABB, BV (BS and others), BSP/Gauss maps, other support data structures
  
physics.Tick:
  update AABBs
  find AABBs intersections
  refine AABBs intersections with BV tests (optional)
  separating axes collision detection
  resolve physics constraints
    contacts
    friction
    distance
    ...
  apply kinematics
\end{lstlisting}
\end{frame}

\section{Assignment}
\begin{slide}{Assignment}{Assignment}{
\item Before the end of next week
\item Group-work archive/video on Natschool or uploaded somewhere else and linked in your report
\item Individual report by each of you on Natschool
\item Build a kinematics simulator with movement and rotation, with RK2 or RK4
}\end{slide}

\begin{frame}{That's it}
\center
\fontsize{18pt}{7.2}\selectfont
Thank you!
\end{frame}

\end{document}


\begin{slide}{SECTION}{SLIDE}{
\item i
}\end{slide}

\begin{frame}[fragile]{SLIDE}
\begin{lstlisting}
CODE
\end{lstlisting}
\end{frame}

\begin{frame}{SLIDE}
\center
%\includegraphics[height=5cm]{Pics/recursive_multiplier.png}
\end{frame}
