\documentclass{beamer}
\usetheme{Warsaw}
\usepackage{nhtvslides}
\usepackage{graphicx}
\usepackage{listings}
\lstset{language=CAML,
basicstyle=\ttfamily\footnotesize,
frame=shadowbox,
breaklines=true}
\usepackage[utf8]{inputenc}

\title{Missing pieces}

\author{Dr. Giuseppe Maggiore}

\institute{NHTV University of Applied Sciences \\ 
Breda, Netherlands}

\date{}

\begin{document}
\maketitle

\begin{slide}{An actual physics engine?}{Points of difference}{
\item How far are we from an actual, commercial engine?
\item Turns out, not very
\item There are some topics we did not touch
\item Some others that we did not deepen enough
\item All in all, what you have seen so far is the hard core of a physics engine
}\end{slide}

\begin{slide}{An actual physics engine?}{Points of difference}{
\item GJK vs SAT
\pause
\item Constant rotational velocity
\pause
\item SAT with BSP for the meshes
\pause
\item Inertia tensor derivation
\begin{itemize}
\item For known shapes
\item For arbitrary meshes
\end{itemize}
\pause
\item Inverse kinematics
\begin{itemize}
\item Standalone
\item Within the collision response system
\end{itemize}
}\end{slide}

\begin{frame}{That's it}
\begin{block}{\center The course is over}
\center
\fontsize{18pt}{7.2}\selectfont
Thank you!
\end{block}
\end{frame}

\end{document}


\begin{slide}{SECTION}{SLIDE}{
\item i
}\end{slide}

\begin{frame}[fragile]{SLIDE}
\begin{lstlisting}
CODE
\end{lstlisting}
\end{frame}

\begin{frame}{SLIDE}
\center
%\includegraphics[height=5cm]{Pics/recursive_multiplier.png}
\end{frame}
